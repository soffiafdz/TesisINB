El abuso de drogas ilícitas es uno de los problemas de salud pública más
importantes del país. El consumo de sustancias produce alteraciones plásticas en el
cerebro que pueden desencadenar consecuencias graves para la salud y funcionalidad
social de los consumidores. La estimulación magnética transcraneal repetitiva
(EMTr) ha surgido como un posible tratamiento para manejar el \textit{craving}
\footnote{Sensación de deseo intenso hacia el consumo de la sustancia.},
la impulsividad y la sintomatología afectiva de la adicción. \par
No obstante, son limitados aún los estudios que comprueben su eficacia clínica
como tratamiento. Y en menor cantidad aún, aquellos que complementan la exploración
 por medio de técnicas de neuroimagen. \par
Se propone en el presente proyecto, un estudio longitudinal
doble ciego, donde se evalúan los efectos clínicos del tratamiento por EMTr y los
cambios neuroplásticos por medio del análisis de neuroimagen funcional y el estudio
de redes complejas o Teoría de Grafos.
