El abuso de drogas ilícitas es uno de los problemas de salud pública más
importantes del país y del mundo.
El consumo de sustancias produce alteraciones plásticas en el
cerebro que pueden desencadenar consecuencias graves para la salud y funcionalidad
social de los consumidores. \par
A pesar de esto, el tratamiento existente para la dependencia a las sustancias es insuficiente e ineficiente.
Con un enfoque centrado en listas de síntomas y patrones de consumo, contados tratamientos farmacológicos aprobados por la FDA
\footnote{Ninguno para la dependencia a cocaína.}
y una tasa de recaída cercana al 50\%, la búsqueda de nuevos y mejores enfoques de tratamiento es imperativa.\par
Es necesario aprovechar el avance en el entendimiento de los mecanismos neurobiológicos del cerebro adicto para el establecimiento de biomarcadores de la adicción y desarrollo de estrategias más eficientes de tratamiento.\par
Uno de estos nuevos enfoques es la estimulación magnética transcraneal repetitiva (EMTr), donde por medio de pulsos electromagnéticos que estimulan el funcionamiento de la corteza prefrontal, permite un mejor manejo del \textit{craving}
\footnote{Sensación de deseo intenso hacia el consumo de la sustancia.}
, la impulsividad y la sintomatología afectiva de la adicción. \par
No obstante, son limitados aún los estudios que comprueben su eficacia clínica como tratamiento, con diseños exploratorios o faltos de rigurosidad y muestras pequeñas. \par
Se propone entonces, en el presente proyecto, un estudio de diseño mixto donde se amplien los conocimientos actuales sobre las bases neurobiológicas de la adicción al mismo tiempo que se compruebe la eficacia de la estimulación magnética transcraneal repetitiva como un tratamiento para la dependencia a la cocaína.\par
Primero, en un diseño transversal se comparan las redes de conectividad funcional de pacientes dependientes a cocaína con las de un grupo pareado de controles sanos con el fin de ampliar los conocimientos actuales sobre la organización topológica del cerebro adicto.\par
En este análisis observamos diferencias en la topología global del cerebro adicto, en donde este se encuentra hiperconectado pero con menores índices de cualidad de pequeño mundo que aquellas redes de sujetos controles sanos.
Bajo un diseño longitudinal monocéntrico a doble ciego, evaluamos los efectos clínicos del tratamiento de estimulación magnética transcreneal repetitiva a \SI{5}{\hertz} sobre la corteza prefrontal dorsolateral. En este análisis encontramos un efecto significativo de mejoría clínica a dos semanas tanto de la sensación subjetiva de \textit{craving} expresada por medio de una escala visual análoga como de la medida de impulsividad explorada por la escala de impulsividad de Barratt. Notamos diferentes patrones de mejoría relativos al estado basal de los participantes y, aunque esta mejoría clínica se mantuvo a los 3 meses de mantenimiento, en la medición posterior a los seis meses de mantenimiento comenzamos a notar una atenuación e incluso empeoramiento de los efectos clínicos principalmente en aquellos sujetos que comenzaron con una sintomatología disminuida. \par
De igual forma, utilizando las mismas técnicas de neuroimagen funcional y teoría de grafos, exploramos la topología de redes funcionales en las diferentes etapas del estudio para observar los efectos del tratamiento sobre los neurocircuitos cerebrales y asociar la mejoría clínica con los cambios en la neurobiología cerebral. Aunque la asociación con las escalas clínicas no fue tan directa como esperaríamos, notamos un efecto paralélo a la mejoría clínica donde la hiperconectividad disminuía a causa del tratamiento de estimulación magnética del mismo modo que la cualidad de pequeño mundo aumentaba. Estos efectos se mantuvieron en las mediciones posteriores a las sesiones de mantenimiento. \par
Nuestros resultados aportan evidencia sobre la efectividad del tratamiento de estimulación magnética transcraneal en la sensación de \textit{craving} e impulsividad en pacientes adíctos, así como demostrar que los análisis globales cerebrales y aquellos basados en técnicas de teoría de grafos pueden ser adecuados para la exploración de la naturaleza, evolución y tratamiento de la adicción.

