El abuso de drogas ilícitas es un problema social y de salud importante en el país y el mundo.
El consumo de sustancias produce alteraciones plásticas en el
cerebro que pueden desencadenar consecuencias graves para la salud y funcionalidad
social de los consumidores. \par
El tratamiento para la dependencia a las sustancias es insuficiente e ineficiente.
Con un enfoque centrado en listas de síntomas y patrones de consumo, contados tratamientos farmacológicos aprobados por la FDA\footnote{Ninguno para la dependencia a cocaína.}
y una tasa de recaída cercana al 50\%, la búsqueda de nuevos y mejores enfoques de tratamiento es imperativa.\par
Es necesario aprovechar el avance en el entendimiento de los mecanismos neurobiológicos de la adicción para el establecimiento de biomarcadores de la adicción y desarrollo de estrategias más eficientes de tratamiento.\par
Uno de estos nuevos enfoques es la estimulación magnética transcraneal repetitiva (EMTr). Se ha visto que esta, por medio de pulsos electromagnéticos que estimulan el funcionamiento de la corteza prefrontal, permite un mejor manejo del \textit{craving}\footnote{Sensación de deseo intenso hacia el consumo de la sustancia.}, la impulsividad y la sintomatología afectiva de la adicción. \par
No obstante, son limitados aún los estudios que comprueban su eficacia clínica como tratamiento, ya que son realizados con base en diseños exploratorios, carecen de un grupo control o exploración neurobiológica, o bien, sus muestras son muy pequeñas.\par
Proponemos con el presente proyecto un estudio de diseño mixto en donde se amplíen los conocimientos actuales sobre las bases neurobiológicas de la adicción al mismo tiempo que se compruebe la eficacia de la estimulación magnética transcraneal repetitiva como un tratamiento para la dependencia a la cocaína.\par
Primero, debido a la escasa investigación sobre la topología de redes funcionales en pacientes con adicción a cocaína,  tomamos datos independientes de controles sanos de un estudio anterior para explorar las diferencias existentes entre la topología de redes de nuestros pacientes adictos a cocaína antes de iniciar cualquier tratamiento y sujetos sin adicción.\par
En este primer análisis observamos diferencias en la topología global del cerebro de las personas con dependencia a cocaína, encontrando una hiperconexión con menores índices de cualidad de mundo pequeño que aquellas redes de los sujetos controles.\par
Nuestro análisis principal consistió en un diseño longitudinal monocéntrico a doble ciego, donde evaluamos los efectos clínicos del tratamiento de estimulación magnética transcraneal repetitiva a \SI{5}{\hertz} sobre la corteza prefrontal dorsolateral.
En este análisis encontramos un efecto significativo de mejoría clínica a dos semanas tanto de la sensación subjetiva de \textit{craving} expresada por medio de una escala visual análoga como de la medida de impulsividad explorada por la escala de impulsividad de Barratt. Notamos diferentes patrones de mejoría relativos al estado basal de los participantes y, aunque esta mejoría clínica se mantuvo a los tres meses de mantenimiento, en la medición posterior a los seis meses de mantenimiento comenzamos a notar una atenuación e incluso empeoramiento de los efectos clínicos principalmente en aquellos sujetos que comenzaron con una sintomatología disminuida.\par
De igual forma, utilizando las mismas técnicas de neuroimagen funcional y ciencia de redes, exploramos la topología de redes funcionales en las diferentes etapas del estudio para observar los efectos del tratamiento sobre los neurocircuitos cerebrales y asociar la mejoría clínica con los cambios neurobiológicos. Aunque no encontramos una relación estadísticamente significativa entre estos cambios con las escalas clínicas, notamos un efecto paralelo a la mejoría clínica donde la hiperconectividad disminuye a causa del tratamiento de estimulación magnética del mismo modo que la cualidad de mundo pequeño aumentaba. Estos efectos se mantuvieron en las mediciones posteriores a las sesiones de mantenimiento.\par
Nuestros resultados aportan evidencia de la efectividad del tratamiento de estimulación magnética transcraneal sobre la sensación de \textit{craving} y la impulsividad en pacientes adictos, así como demostrar que los análisis globales cerebrales y aquellos basados en técnicas de teoría de grafos pueden ser adecuados para la exploración de la naturaleza, evolución y tratamiento de la adicción.
