% Discusion; DLPFC vs VMPFC
Determinar cual subregion de la corteza prefrontal media cuál función puede ser muy dificil, debido a la flexibilidad neuroanatómica y cognitiva de estas funciones. (La corteza cingulada anterior dorsal y la DLPFC están involucradas en la respuesta al craving, control sobre el craving o ambas?); o sea, participantes puedn usar múltiples estrategias cuando realizan tareas neuropsicológicas, y los sistemas prefrontales muestran un mayor nivel de flexibilidad funcional que los sistemas motores primarios \parencite{Goldstein2012a}.

% Discusion; VMPFC mejor target craving
El rol de la corteza orbitofrontal medial en la supresión del craving fue propuesto por hallazgos en un estudio de PET en usuarios a cocaína; el incremento en \textit{craving} fue relacionado con el metabolismo de glucosa en la corteza prefrontal medial \parencite{Goldstein2012a}

% Discusion; VMPFC craving DLPFC ejecutivo
Se distingue a regiones de la PFC que facilitan los esfuerzos cognitivos no relacionados a la droga y al control inhibitorio (DLPFC, dACC y giro frontal inferior); y aquellas que reflejan preocupación emocional relacionada a la droga, \textit{craving}, y comportamientos compulsivos (mOFC y ACC ventral) \parencite{Goldstein2012a}.

% Discusion; izquierda, derecha, no importa
Los estudios revisados no aportan un patrón claro que indique lateralización de los cambios cerebrales en sujetos con adicción \parencite{Goldstein2012a}

% ???
En adictos a coca, la corteza anterior del cíngulo rostroventral tenia una menor conectividad con el cerebro medio \parencite{Gu2010}

% Discusion; DMN & Craving
Durante el craving, el componente prefrontal de la DMN (que procesa señales interoceptivas) es activado; y durante la inhibición cognitiva, desactivado.
% Justificacion; Estimular PFC
Mediación frontal de un circuito neural involucrado en la respuesta al craving puede funcional como un objetivo para intervenciones cognitivas \textit{top-down}que pueden tener un beneficio terapéutico. Intervenciones que refuercen a un debilitado pero aun funcional circuito fronto-accumbal puede incrementar la habilidad de usuarios a cocaína para bloquear o reducir la respuesta al craving \parencite{Volkow2010a}.

% Justificacion/Discussion; Craving (stress) lleva a relapso
Una respuesta mayor de craving a cocaina inducido por el estrés fue asociado con un menor tiempo al relapso; de igual forma que la cantidad de consumo en la linea base.
Craving a la droga relacionado al estrés y respuestas psicobiologicas son predictivas del subsecuente relapso. Estres > que cue \parencite{Sinha2005}

% Discussion; mejores alternativas a VAS
Debate actual de que escalas simples como la escala visual análoga tienen validez facial pero carecen de validez de contenido y predictiva\parencite{Ekhtiari2018}; la mayoría de las investigaciones la utilizan \parencite{Barr2011}.

% Justificacion; evaluar control con tratamiento cotidiano
Aunque las técnicas neuromoduladoras son una intervención prometedora como tratamiento de la adicción a sustancias, la mayoría de las efectos son parciales, e incluso los efectos contra el \textit{craving} bien documentados de la EMTr no necesariamente significan una reducción en el uso o en abstinencia \parencite{Ekhtiari2019}, por lo que combinar la neuromodulación con intervenciones comportamentales y farmacoterapéuticas puede mitigar estas deficiencias.

% Justificacion/discussion; lateralidad en TMS target
La mayoria de los estudios se enfocan en excitar la DLPFC (siguiendo la tendencia marcada por las investigaciones sobre la EMTr en depresion); no obstante, se ha encontrado que la EMTr lateralizada puede producir cambios bilaterales en los patrones de activacion cerebral, como la activación de aferentes monosinápticos en el hemisferio contralateral \parencite{Hanlon2013}
o influenciando la conectividad funcional en reposo de circuitos frontoestriatales \parencite{Schluter2017}.

% Justificacion; biomarcadores son necesarios
Actualmente no hay biomarcadores clínicos útiles para la adicción a sustancias. Un pobre entendimiento del cerebro humano adicto y de los efectos complejos de las drogas en los distintos mecanismos neurobiológicos y circuitos neuronales contribuyen a la falla de desarrollar tratamientos efectivos, ya que el campo de la psiquiatría principalmente depende/se apoya de listas de síntomas y marcadores de consumo. Para avanzar en el campo, muchos investigadores consideran necesario comenzar a depender de marcadores y medidas cerebrales.

% Justificacion; topologia de red
Para efectivamente diagnosticar y tratar a pacientes con dependencia de sustancias, en vez de enfocarse en una región cerebral o neurotransmisor específico, un mejor entendimiento de cómo la droga abusada afecta la organización topológica y las redes de conectividad cerebral puede tener una mucho mayor importancia estratégica \parencite{Steele2018}.

% Discusion/Justificacion; seleccion de subredes
La trayectoria del desarrollo de la dependencia de drogas de un uso impulsivo a compulsivo se refleja en los cambios de varios constructos cognitivos y sus respectivas redes neuronales: procesamiento de recompensas, atribución de saliencia, control ejecutivo y rumiaciones internas \parencite{Ekhtiari2018}.

% Discussion; en normalidad + Costo / + eficiencia
En \textcite{Achard2007} la eficiencia incrementó monotónicamente como una función del costo en la red.

% Discussion; Small-worldness y Eficiencia se ven afectadas por bloqueo de D2R
En un estudio donde aplicaron un antagonista de receptor de dopamina a dos muestras independientes de adultos jóvenes y mayores, \textcite{Achard2007} llegaron a la conclusión de que las propiedades de pequeño mundo de las redes, tanto la efectividad global como la local, se veían afectadas tanto por la adultez como por el bloqueo de receptores D2 a cocaína. Este efecto dopaminérgico fue altamente localizado en el cíngulo dorsal y la corteza temporal lateral.

% Discussion; explicacion de hiperconectividad
La paradoja de tenero un incremento en la fuerza de conexión pero una reducción en las métricas de red puede deberse a que la fuerza de conectividad elevada sea compnensatoria a la pérdida de eficiencia \parencite{Wang2015a}.

% Figures



