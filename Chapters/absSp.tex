\selectlanguage{spanish}
\begin{center}
    \large\textbf{Conectividad Cerebral en Pacientes Adictos a Cocaína después de
     Tratamiento con Estimulación Magnética Transcraneal Repetitiva}
\end{center}
\begin{quotation}
    \noindent
    Aún hay poca e inadecuada investigación sobre la efectividad de la estimulación magnética transcraneal repetitiva (EMTr) como tratamiento para la dependencia a cocaína. En este estudio clínico longitudinal, mono-céntrico, doble-ciego exploramos las diferencias en topología entre pacientes con adicción a cocaína y controles sanos y usamos estos datos para evaluar los cambios clínicos y de topología después de un tratamiento de EMTr sobre la corteza prefrontal dorsolateral. La topología global y escalas clínicas fueron medidas en 40 participantes antes y después de recibir dos sesiones diarias de EMTr real o sham por dos semanas; así como después de recibir sesiones semanales de mantenimiento por tres ($n=16$) y seis meses ($n=11$). Nuestro análisis preliminar mostró diferencias significativas en el costo, eficiencia y cualidad de pequeño-mundo entre las redes de nuestros pacientes y sujetos controles. Modelos de efectos mixtos mostraron una interacción significativa entre el grupo de estimulación y la fase de tratamiento tanto para el \textit{craving} como impulsividad. Hubo también cambios significativos en el costo y la métrica de pequeño mundo de las redes atribuibles al tratamiento de EMTr. Todos estos cambios se mantuvieron después de los tres meses de mantenimiento y no fue hasta los seis meses de mantenimiento que empezaron a mostrar un decaimiento. Estos resultados proveen evidencia de la eficacia de la EMTr como una alternativa de tratamiento en la adicción así como la posibilidad de utilizar metodología de teoría de grados para la exploración de la naturaleza, evolución y tratamiento de la adicción.
\end{quotation}
\clearpage
