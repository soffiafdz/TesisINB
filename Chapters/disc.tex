\section{Exploración con controles}
En nuestro análisis preliminar de comparación de nuestra muestra de pacientes con adicción diagnósticada a la cocaína y controles sanos independientes encontramos diferencias significativas entre ambas poblaciones y resultados similares a los reportados previamente por \textcite{Wang2015a} en donde nuestros pacientes demostraron tener un cerebro en reposo hiperconectado. Aunque en el estudio de \textcite{Wang2015a} los cerebros de los pacientes dependientes a substancias eran además menor eficientemente conectados, tanto en las modalidades local y global de eficiencia, como en la métrica de pequeño mundo; nuestros controles sanos resultaron tener medidas más altas en ambas modalidades de eficiencia a lo largo de los umbrales explorados. La métrica de pequeño mundo, por su parte, mostró un patrón consistente con el estudio anterior. \par
La incongruencia en los resultados de eficiencia entre los resultados podría deberse a la diferencia en métodología utilizada, específicamente la naturaleza de las redes exploradas (binarias vs ponderadas) y la estrategia de umbralización \parencite{Hallquist2018}. \par
La escalar de pequeño mundo, al ser una medición obtenida de la comparación entre los datos primarios y datos aleatorios generados a partir de estos, podría ser más resistente a estas variaciones derivadas de la metodología. \par
Esta métrica está relacionada a la eficiencia de comunicación tanto global como local \parencite{Latora2001}. La reducida escalar de pequeño mundo en nuestros pacientes adictos podría sugerir entonces una comunicación inter-regional alterada que podría ser base de la pérdida de control cognitivo e inhibición característicos de la adicción. La paradoja de tener un cerebro hiperconectado pero una reducción en esta medida de eficiencia podría deberse a que la fuerza y densidad de conexión elevada sea una medida compensatoria a la comunicación disfuncional.

\section{Fase cerrada}
\subsection{Mejoría clínica}
En la parte de ensayo clínico del proyecto observamos una relación basal en ambas modalidades de la misma prueba (versiones \textit{Now} y \textit{General} del CCQ), y entre pruebas que medían lo mismo (VAS y CCQ-N; \textit{craving} en el momento) lo que era de esperarse. No obstante, una vez pasadas las dos semanas de tratamiento encontramos una correlación incrementada entre las mediciones clínicas lo que sería un primer indicio del efecto del tratamiento. \par
Nuestros resultados clínicos fueron consistentes con estudios anteriores que demuestran una mejoría en la sensación de \textit{craving} (medida por VAS) posterior a un tratamiento de estimulación magnética transcraneal sobre la corteza prefrontal dorsolateral \parencite{Politi2008, Terraneo2016}. \par
Además del efecto de la membresía al grupo experimental, notamos un patrón de mejoría distinto relacionado al estado basal clínico de los pacientes antes del inicio al tratamiento en la medición de \textit{craving} al momento (VAS y CCQ-N) que no había sido reportado en la literatura.
El tratamiento fue más efectivo en estas mediciones (incluyendo el efecto de tratamiento habitual/placebo en aquellos participantes del grupo sham) en los sujetos que comenzaron la investigación con niveles más altos de \textit{craving}. Los participantes que iniciaron la investigación con niveles clínicos basales más leves mostraron un patrón de cambio inverso, manteniendose constantes o empeorando al pasar las dos semanas de tratamiento placebo.
En nuestra exploración estadística observamos que la mejoría, controlando por grupo experimental y niveles basales clínicos (tanto de \textit{craving} como de impulsividad) fue significativa en las mediciones de impulsividad y de \textit{craving} por la escala visual análoga. En cuanto a las otras mediciones de \textit{craving}, el cuestionario  CCQ-N, a pesar de mostrar un patrón similar a la VAS y estar correlacionada con esta, no alcanzó los niveles de significancia estadística; el cuestionario CCQ-G, en cambio, mostró estar mayormente influenciada por los niveles de impulsividad que las demás escalas de \textit{craving}.
Podemos concluir con seguridad entonces que, controlando por los niveles basales clínicos, nuestro tratamiento de dos sesiones diarias de estimulación magnética transcraneal repetitiva por dos semanas es clínicamente efectivo en el manejo de \textit{craving} e impulsividad en comparación con el tratamiento habitual.

\subsection{Topología de redes}
En los cambios de topología a las dos semanas de tratamiento pudimos observar, de la misma forma, un efecto significativo atruibuible al tratamiento de estimulación magnética. En las medidas topológicas, con las dos semanas de estimulación transcraneal repetitiva, los pacientes mostraron una reducción en el costo de sus redes tanto en la cantidad de conexiones de las mismas (densidad) como la fuerza de estas. Este cambio no fue observable en aquellos participantes cuya estimulación fue simulada, quienes mostraron en cambio un incremento (aunque este no fue estadísticamente significativo).\par
En las mediciones de eficiencia, observamos un patrón similar en la medición de eficiencia global, donde la disminución de esta métrica es atribuible al tratamiento de estimulación. Aunque este resultado va en contra de nuestra hipótesis inicial y lo marcado en la literatura \parencite{Wang2015a} es consistente con lo hallado en nuestro análisis preliminar de comparación con una muestra de controles sanos. \par
De igual forma que en el análisis transversal se encontró una inconsistencia en las medidas de eficiencia y la métrica de pequeño mundo, nuestros pacientes pertenecientes al grupo de estimulación real, a pesar de haber obtenido una disminución en sus medidas de eficiencia, tuvieron un aumento en la cualidad de pequeño mundo de sus redes de conectividad atribuible al tratamiento de estimulación, mientras que los participantes que recibieron estimulación simulada mostraron un decremento en la misma medición. \par
Aunque estos resultados fueron obtenidos controlando por las mediciones clínicas, estas no fueron una covariante significativa. Del mismo modo, a pesar de que no podemos demostrar una relación entre la mejoría clínica y estos cambios en la topología funcional de las redes, nuestros hallazgos sugieren que los efectos del tratamiento por estimulación magnética transcraneal son a partir de los cambios producidos sobre la reorganización de las redes de conectividad funcional en reposo.

\section{Fase abierta (mantenimiento)}
Ningún estudio de la estimulación magnética transcraneal repetitiva como tratamiento a la adicción ha explorado los efectos de mantenimiento a largo plazo.\par
\textcite{Terraneo2016} exploraron mantenimiento hasta a 3 semanas y, aunque \textcite{Bolloni2016} exploraron los efectos posteriores a tres y seis meses, estas mediciones de seguimiento  fueron sin contar con sesiones de EMTr de mantenimiento.\par
En las cuatro mediciones clínicas observamos que, aunque no hubo una mejoría con respecto a las dos semanas de tratamiento, la mejoría clínica se mantuvo constante con las sesiones semanales de mantenimiento.
Sin embargo, al observar los cambios a los seis meses de mantenimiento nos encontramos con un ligero empeoramiento en comparación con T2 (tres meses de mantenimiento). Este empeoramiento fue principalmente en aquellos sujetos que comenzaron con puntajes basales bajos, volviendo a niveles basales o incluso peores (VAS). Para aquellos participantes que comenzaron con niveles clínicos altos y la efectividad del tratamiento fue mayor, este efecto de empeoramiento fue menor (VAS, CCQ-G) y en la impulsividad fue inexistente.

Los estudios de topología funcional con análisis de teoría de grafos en la adicción son aún escasos; los análisis longitudinales en este tipo de muestras clínicas con esta metodología son inexistentes. Es por esto que nuestros análisis fueron de naturaleza exploratoria.\par
No hallamos un patrón claro en los cambios observados longitudinalmente en las métricas de topología. Similar que con las mediciones clínicas nos encontramos con un mantenimiento del efecto obtenido con el tratamiento tanto en el costo de las redes como en su eficiencia. En la cualidad de pequeño mundo observamos un ligero decremento que no fue significativo, por lo que el efecto también se mantuvo constante. \par
Sin embargo, cuando exploramos las mismas medidas posterior a los seis meses de mantenimiento observamos que hubo de nuevo una disminución en las medidas de densidad y de fuerza yendo incluso por debajo de los efectos obtenidos con el tratamiento. Lo mismo sucedió con la modalidad de eficiencia local, mas no global. En la métrica de pequeño mundo observamos un cambio de signo en el cambio comparado con el observado a los tres meses. Al explorar las métricas de topología de las que esta medida deriva, observamos que el largo de camino vuelve a aumentar posterior a la medición de los tres meses y el coeficiente de agrupamiento disminuye a niveles basales. \par

\section{Alcances y limitaciones}
Aunque ya ha habido estudios anteriores que exploran la eficacia de la estimulación transcraneal repetitiva como un tratamiento efectivo para la dependencia a la cocaína, este es el primero que explora por medio de un doble-ciego, con una muestra de más de 40 sujetos y además analiza los efectos neurobiológicos por medio de una exploración con base en imagen por resonancia magnética. \par
Nuestros resultados nos permiten defender la idea de que la EMTr es una posible alternativa de tratamiento para la dependencia a la cocaína y que sus efectos clínicos pueden mantenerse a largo plazo por medio de sesiones semanales de mantenimiento. \par
Entre nuestras limitaciones se encuentra el alto grado de deserción de nuestros pacientes, lo que influyó el tamaño de nuestra muestra en los análisis longitudinales. Esto podría haber afectado nuestra exploración topológica y dado lugar a los hallazgos inexplicables. \par
Similarmente, los hallazgos clínicos no fueron generalizables a lo largo de las distintas escalas utilizadas. A pesar del debate sobre la carencia de validez de contenido y predictiva de la VAS \parencite{Ekhtiari2019}, fue con esta medición donde encontramos una mayor relación con la mejoría de la impulsividad y fue más claro el efecto del tratamiento. Nuestros sujetos comentaron a los aplicadores frecuentemente que no comprendían totalmente las preguntas de los cuestionarios CCQ. \par
Sería útil para futuras investigaciones utilizar medidas más objetivas para el \textit{craving} y la recaída, ya que nuestra investigación se basa principalmente en mediciones auto-reportadas. \par
Los parámetros utilizados en nuestro paradígma de estimulación podrían ser mejores. Futuras investigaciones podrían explorar la diferencia entre frecuencias más altas de estimulación o ubicaciones distintas a la corteza prefrontal dorsolateral.\par
Nuestro trabajo deja las bases para continuar la exploración tanto de la estimulación magnética transcraneal como un tratamiento efectivo, así como sus efectos a largo plazo y la utilización de metodología de teoría de grafos para la investigación de bases biológicas de la adicción, su desarrollo y evolución. Los resultados sobre los diferentes patrones de mejoría de acuerdo a los puntajes basales dejan claro que el siguiente paso es explorar biomarcadores (ya sea clínicos o de neuroimagen) que propicien un mejor efecto del tratamiento.
