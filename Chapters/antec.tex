\section{Neurobiología de la adicción}
Anteriormente, persistían las ideas de que el abuso de sustancias era un acto voluntario y hedonista.
No obstante, la investigación de las últimas décadas ha venido apoyando la idea de que la adicción es una enfermedad del cerebro \parencite{Volkow2016}.\par
El modelo más reciente de la adicción la describe como un síndrome de inhibición de respuesta y atribución de salienta dañados \parencite{Goldstein2012a}, compuesto por conductas impulsivas y compulsivas \parencite{Koob2010a} y que se caracteriza por:
\begin{enumerate}
    \item{compulsión en buscar y consumir la sustancia; }
    \item{pérdida de control limitando el consumo, y; }
    \item{emergencia de un estado emocional negativo reflejando un síndrome motivacional de abstinencia.}
\end{enumerate}
La adicción, entonces, en su estudio suele dividirse en tres estados: intoxicación, abstinencia y afecto negativo, y, preocupación y anticipación (o \textit{craving}) \parencite{Koob2010a,Goldstein2012a,Volkow2016}.

\subsection{Intoxicación}
\label{intox}
Todas las drogas adictivas activan regiones cerebrales de recompensa en el cerebro que producen incrementos en la liberación de dopamina, lo que a su vez genera un aprendizaje asociativo o condicionamiento\footnote{En este tipo de aprendizaje Pavloviano, las experiencias recompensantes repetidas terminan asociándose con los estímulos mentales que les preceden.}.
Ante repetidas exposiciones, las neuronas dopaminérgicas dejan de disparar ante la droga en sí y responden de forma anticipatoria ante los estímulos condicionados. Esta liberación de dopamina desencadena neuroplasticidad\footnote{Cambios sistemáticos en la señalización o comunicación sináptica entre las neuronas.}, tanto como potenciación a largo plazo \textemdash donde la transmisión de señales entre las neuronas aumenta\textemdash{} como depresión a largo plazo \textemdash donde la señalización disminuye. Estos cambios en fuerza sináptica son controlados por la inserción o retiro de receptores glutamatérgicos (AMPA y NMDA) y cambios en la composición de sus subunidades \parencite{Volkow2016}. La regulación de los receptores AMPA incrementa la capacidad de respuesta del núcleo accumbens a glutamato libreado por terminales corticales y límbicas ante la exposición a drogas o estímulos relacionados \parencite{Wolf2010}. Cambios neuroplásticos han sido observados también en el estriado dorsal, la amígdala, el hipocampo y la corteza prefrontal \parencite{Volkow2016}.\par
Los estímulos ambientales relacionados al consumo, mediante este mecanismo, desencadenan una liberación rápida y condicionada de dopamina que provoca un antojo (o \textit{craving}) por la droga \parencite{Volkow2006}, motivan conductas relacionadas a la búsqueda de la sustancia y llevan a un uso compulsivo (atracón) \parencite{Volkow2016}.

\subsection{Abstinencia}
\label{abst}
Como resultado de los procesos fisiológicos condicionados, las recompensas ordinarias y saludables pierden su poder motivacional.
En las personas adictas se desarrolla un sesgo motivacional que provoca que los sistemas motivacionales y de recompensa se concentren el la liberación más potente de dopamina producida por la sustancia \textemdash{} y los estímulos ambientales condicionados \parencite{Volkow2016}.\par
Contrario a lo que se podría pensar, estudios clínicos y pre-clínicos han demostrado que el consumo de sustancias produce incrementos mucho más pequeños de los niveles de dopamina en la presencia de adicción \parencite{Volkow1997,Zhang2013,Volkow2014}.
Esto deja al sistema de recompensa del cerebro mucho menos sensible a estimulación para estímulos tanto relacionados- como no-relacionados a la sustancia; y como resultado, las personas con adicción no experimentan el mismo grado de euforia producida por el consumo al mismo tiempo que se sienten menos motivados por los estímulos de la vida diaria
\footnote{Estos cambios no son inmediatamente reversibles e.g. desintoxicación.} \parencite{Volkow2016}.\par
Ante una exposición repetida a los efectos liberadores de dopamina de las sustancias se lleva a adaptaciones en el circuito extendido de la amígdala, lo que resulta en un incremento en la reactividad ante el estrés y la emergencia de emociones negativas.
De esta forma, además de la motivación por la "recompensa" producida por el consumo, hay una motivación intensa a escapar de la disforia asociada con los efectos posteriores al uso \parencite{Goldstein2012a,Volkow2016}.
Este fenómeno está relacionado con la hipótesis de procesos-oponentes de \textcite{Solomon1978} que describe la dinámica temporal de respuestas emocionales opuestas.
En este modelo se observa cuando el refuerzo negativo (escapar de la disforia de la abstinencia) prevalece sobre el refuerzo positivo (la búsqueda del \textit{high} en la intoxicación aguda, en la transición del abuso ocacional de la sustancia al desarrollo de la adicción.
Desafortunadamente, aunque los efectos breves de incremento de dopamina posteriores al consumo alivian temporalmente la aflicción, el resultado del consumo repetido es un incremento en la disforia durante la abstinencia, lo que lleva a un ciclo vicioso.

\subsection{Preocupación y anticipación}
\label{crav}
Los cambios que ocurren en los circuitos emocionales y de recompensa son acompañados por cambios en la función de regiones corticales prefrontales.
Entre los efectos de las alteraciones a estas regiones están la perturbación de procesos ejecutivos\footnote{Capacidades de auto-regulación, toma de decisiones, flexibilidad en la selección e inicio de la acción.}, atribución de saliencia y el monitoreo del error \parencite{Goldstein2012a,Volkow2016}.
En sujetos adictos, la señalización afectada de dopamina y glutamato en estas regiones prefrontales debilita la habilidad de resistir deseos fuertes o de seguir con la decisión de dejar el consumo de la sustancia \parencite{Volkow2016}.\par
El \textit{craving}, o el deseo fuerte e intenso de consumir la sustancia, tanto para volver a sentir los efectos eufóricos como para evitar los aspectos de abstinencia provocados por su ausencia, es un elemento clave en la recaída \parencite{Koob2010a}.
El \textit{craving} es entonces, por estas características, un punto clave como enfoque para el tratamiento de las adicciones.

\section{Tratamiento para la adicción a cocaína}
Actualmente no existe una cura para la adicción.
El tratamiento consiste en un abordaje multidisciplinario que ayuda al manejo de la enfermedad.
Generalmente este consiste en el apoyo psicoterapéutico, consejería y tratamiento farmacológico en conjunto. \par
No obstante, la naturaleza crónica de la adicción hace que la recaída no sea solo posible, sino probable \parencite{NIDA.}.
Estudios de seguimiento a 1 año postratamiento\footnote{Datos de tratamientos en EEUU.} han encontrado que solo del 40-60\% se mantienen en abstinencia \parencite{McLellan1980}.\par
Debido a esto es de suma importancia la revisión del manejo actual de la adicción y la búsqueda de nuevos tratamientos alternativos que sean más efectivos.

\subsection{Tratamiento farmacológico}
No existe actualmente un tratamiento farmacológico aprobado por la FDA (Administración de comida y drogas, por sus siglas en inglés) de Estados Unidos para la dependencia a la cocaína.
Los medicamentos utilizados comúnmente son los mismos que aquellos usados para tratar la epilepsia o los espasmos musculares, principalmente con el fin de aliviar la ansiedad y la agitación resultantes de la adicción a la cocaína.
Algunos de estos son: gabapentina, un anticonvulsivante análogo a GABA; modafinilo, promueve el estado de alerta inhibiendo la re-captura de dopamina; topiramato, anticonvulsivante que alivia la agitación; vigabatrina, anti-epiléptico usado para el \textit{craving} inhibiendo el catabolismo de GABA; y, baclofeno, relajante muscular agonista de GABA \parencite{Volkow2007b}.\par
De todos los medicamentos usados para tratar la adicción a la cocaína, disulfiram es el que ha tenido más resultados exitosos consistentes \parencite{Volkow2007b}.
Usualmente utilizado como tratamiento a la adicción al alcohol por medio de la inducción de una reacción adversa al alcohol, disulfiram también a sido prescrito para desalentar el uso de la cocaína.
Aún no se conoce específicamente como funciona, pero sus efectos pueden estar relacionados a su capacidad de inhibir una enzima que convierte a la dopamina en noradrenalina.
Sin embargo, no es efectivo para todas las personas, ya que se ha visto que ciertas variaciones genéticas influyen en la efectividad del tratamiento \parencite{Gaval-Cruz2009a, Volkow2007b}.

\subsection{Tratamiento comportamental}
Junto al tratamiento farmacológico, usualmente se lleva a cabo un abordaje comportamental por medio de psicoterapia y consejería.
Entre los modelos más utilizados dentro de estas áreas están:
la psicoterapia cognitivo-conductual, donde el paciente aprende a identificar y corregir comportamientos problemáticos aplicando una serie de técnicas que pueden ser usadas para detener el abuso de la sustancia y abordar los problemas que suelen ocurrir en conjunto;
el manejo de contingencias, donde por medio de recompensas tangibles se refuerzan los comportamientos positivos como la abstinencia;
intervención motivacional, es un abordaje de consejería donde se ayuda al paciente a resolver su ambivalencia sobre el compromiso al tratamiento y parar el consumo de la sustancia;
y la terapia familiar conductual, que intenta abordar no solo el consumo de la droga, sino también los problemas en conjunto, como trastornos de conducta, maltrato infantil, depresión, conflictos familiares y desempleo \parencite{Volkow2008}.

\section{Estimulación magnética transcraneal repetitiva}
A pesar de ser originalmente desarrollada como una herramienta diagnóstica, la estimulación magnética transcraneal (EMT) puede modular, transitoria o duraderamente la excitabilidad cortical por medio de la aplicación de pulsos electromagnéticos localizados que pasan de forma indolora y sin daño por la piel y el cráneo \parencite{Horvath2011a, Noohi2016}.
La EMT puede ser puede ser aplicada por medio de pulsos individuales, pulsos pareados, o por medio de repetidos trenes de estimulación, contínuos o a una frecuencia específica (repetitiva o EMTr) o bajo un patrón de intervalos inter-trenes específico (estimulación \textit{theta-burst} contínua, ETBc; o intermitente, ETBi) \parencite{Ekhtiari2018}. \par
Se ha observado que la EMTr a alta frecuencia (\deactivatequoting\SI{> 5}{\hertz}\activatequoting) facilita la excitabilidad cortical-motora, mientras que a baja frecuencia (\deactivatequoting\SI{< 1}{\hertz}\activatequoting) se inhibe \parencite{Pascual-Leone1994}.
De la misma forma, la estimulación con \textit{theta burst} (ETB) presenta patrones similares de estimulación e inhibición cortical en sus modalidades intermitente y contínua respectivamente, pero con una duración menor que la EMTr. El mecanismo primario hipotetizado bajo los efectos neuromodulatorios de ambas técnicas es la potencialización a largo plazo (LTP) y depresión a largo plazo (LTD).
Un rápido incremento post-sináptico de iones de calcio puede inducir LTP, lo que se ha observado en EMTr de alta frecuencia (\SI{10}{\hertz}) y en ETBi; mientras que un flujo mas lento y sostenido de calcio induce LTD en EMTr de (\SI{1}{\hertz}) y ETBc \parencite{Ekhtiari2018}.

\subsection{EMTr como tratamiento en adicción}
Las investigaciones recientes apoyan la habilidad de la EMTr de reducinr el \textit{craving} de tabaco, alcohol y cocaina en pacientes adictos \parencite{Barr2011}.
La mayoría de los estudios clínicos en adicción se enfocan en estimular la corteza dorsolateral prefrontal izquierda \parencite{Bellamoli2014a,Barr2011,Ekhtiari2019}.
Varias líneas de evidencia sugieren que la estimulación sobre esta región puede influir en las regiones cerebrales involucradas en la adicción:
\begin{enumerate}
    \item La corteza prefrontal dorsolateral está conectada al sistema dopaminérgico meso-fronto-límbico, el sistema cererbal de recompensa asociado al \textit{craving} \parencite{Barr2011}.
    \item Se ha demostrado la capacidad de la EMTr de inducir liberación de dopamina en áreas corticales y subcorticales\parencite{Cho2009,Strafella2001}, lo que podría mitigar la disfunción dopaminérgica asociada a la adicción.
    \item El nivel de \textit{craving} a comida, alcohol y tabaco en la presencia de estímulos visuales se ha reducido con la estimulación de la corteza prefrontal dorsolateral con EMTr \parencite{Amiaz2009}.
    \item Se ha encontrado que la corteza prefrontal dorsolateral está involucrada con procesos de toma de decisiones; los cuales pueden estar alterados en pacientes adictos, quienes tienen mayor probabilidad a ser impulsivos y tener comportamniento asociado a la toma de riesgos \parencite{Barr2011}.
\end{enumerate}
Las características neuromoduladoras y los cambios plásticos observados como efecto de la EMTr hacen de esta un tratamiento efectivo factible para alterar los circuitos involucrados en el \textit{craving} de la adicción, lo que, por efecto, llevaría a la disminución de la recaída y una mayor efectividad de tratamiento.

\subsubsection{EMTr y adicción a cocaína}
El consumo crónico de cocaína está asociado con una disrupción en la conectividad fronto-estriatal en estado de reposo; una elevada actividad en la corteza prefrontal medial y estriado ventral, y una actividad deprimida en la corteza prefrontal dorsolateral y estriado dorsal, así como una disminución en la transmisión dopaminérgica mesolímbica, lo que mantiene el consumo de la droga \parencite{Rachid2018}.
Estas afectaciones tienen un rol importante en el desarrollo de la adicción, lo que se amplifica y soporta por una incontrolable necesidad de consumir la substancia (\textit{craving}), lo que conduce a su busqueda y relapso.
En efecto, se ha encontrado que mayores indices de \textit{craving} se relacionan con índices más altos de relapso \parencite{Sinha2006,Volkow2000a,Volkow2016}.
Por esto mismo, investigaciones recientes han buscado explorar los efectos terapeuticos de la estimulación de la corteza prefrontal  por medio de EMTr en la adicción a cocaína \parencite{Bolloni2018}. \par
\textcite{Camprodon2007} fueron los primeros en explorar los efectos de la EMTr en el \textit{craving} a cocaína.
Encontraron un efecto transitorio en el \textit{craving} a cocaína después de una sesión de EMTr sobre la corteza prefrontal dorsolateral derecha \textemdash{}pero no la izquierda.
Desde entonces, diversos estudios han encontrado que la EMTr sobre la corteza prefrontal izquierda es capaz de reducir el craving.
\textcite{Politi2008} observaron que después de 10 sesiones de EMTr a \SI{15}{\hertz} había una disminución del \textit{craving} en 36 sujetos con dependencia a cocaína, encontrando un efecto después de la 7ma sesión.
\textcite{Terraneo2016}, en su estudio aleatorio abierto, compararon los efectos terapéuticos sobre la adicción a cocaína entre 8 sesiones de EMTr a \SI{15}{\hertz} (sesiones diarias por 5 días y una sesión semanal de mantenimiento por 3 semanas) y un tratamiento farmacológico habitual.
Encontraron que había una mayor disminución del \textit{craving} en los sujetos del grupo de EMTr comparados con el grupo control, así como una diferencia significativa en la cantidad de pacientes que tuvieron un relapso\footnote{El consumo fue medido por medio de pruebas objetivas de orina}. \par
Aunque la mayoría de las investigaciones de EMTr en adicción se enfocan en la corteza prefrontal dorsolateral, estudios recientes han explorado los efectos sobre la estimulación de la corteza prefrontal ventromedial \parencite{Hanlon2015,Kearney-Ramos2018a}, argumentando que la corteza prefrontal medial es un método más directo de modular la actividad del núcleo accumbens\footnote{Las cortezas prefrontales orbital y medial son la principal entrada cortical al estriado ventral, que incluye al caudado y núcleo accumbens}.
En un estudio piloto experimental, \textcite{Hanlon2015} obtuvieron datos de neuroimagen y de \textit{craving} antes y después de ETB contínua sobre la corteza medial prefrontal izquierda y encontraron que una sesion de ETB real reducía el \textit{craving} comparado con la sesi placebo, así como una reducción de actividad sobre la corteza prefrontal y el núcleo accumbens.
\textcite{Kearney-Ramos2018a} exploraron los efectos de la ETB contínua en la reactividad a estímulos relacionados a la sustancia; encontraron que efecto general de la ETBc atenuando el \textit{craving} después de la sesión real en comparación con la placebo, en usuarios a cocaína. \par
Otro enfoque de tratamiento utilizado recientemente en la EMTr es el empleo de la bobina H1.
Esta bobina permite la estimulación bilateral de regiones más profundas.
\textcite{Bolloni2016} observaron una disminución significativa en el consumo de cocaína en adictos a cocaína a los 3 y 6 meses posteriores a un tratamiento real de EMTr profunda de 12 sesiones (tres a la semana) a \SI{10}{\hertz} sobre la corteza prefrontal bilateral\footnote{La interacción primaria (Tratamiento X Tiempo) fue no significativa en el análisis primario.} que no se encontró en el grupo placebo.
Por su parte, el equipo de \textcite{Rapinesi2016} utilizando la misma técnica de EMTr profunda (pero con una frecuencia mayor, tres sesiones semanales a \SI{15}{\hertz} por cuatro semanas), encontraron una disminución significativa de \textit{craving} en sujetos con dependencia a la cocaína dos, cuatro y ocho semanas después del tratamiento.
No obstante, hubo un empeoramiento en los niveles de \textit{craving} después de la cuarta semana, aunque estos se mantuvieron por debajo de los niveles basales.

\section{Imagenología por resonancia magnética}
A pesar de que la imagenología radioisotópica (PET) ha sido altamente significativa en el descubrimiento y mapeo del rol de la dopamina en la adicción, la imagenología por resonancia magnética (MRI) es el pilar de la investigación con neuroimagen en la adicción, debido a su seguridad, ausencia de radioactividad y flexibilidad en la información obtenida \parencite{Suckling2017}. \par
La MRI puede ofrecer información tanto anatómica \textemdash{}relacionada a la estructura de la materia gris y blanca\textemdash{} como funcional \textemdash{}relacionada a la actividad cerebral.

\subsection{Resonancia magnética funcional}
La resonancia magnética funcional (fMRI) es una técnica de neuroimagen que mide la actividad cerebral por medio del contraste endógeno del nivel dependiente de oxigenación de sangre (BOLD) \parencite{Ogawa1993}, utilizando las distintas propiedades magnéticas de la sangre oxigenada y deoxigenada.
Partiendo de que la actividad neuronal requiere oxígeno, la señal BOLD está indirectamente relacionada con el procesamiento funcional local.
Los experimentos de fMRI, entonces, frecuentemente inducen procesos cognitivos específicos con los estímulos apropiados, con la intención de observar las regiones y circuitos involucrados. \par
Mucho de lo que actualmente se conoce sobre función cerebral ha venido de estudios que miden los cambios en actividad neuronal y conducta después de la administración de una tarea o estímulo (\textit{task-based fMRI}), sin embargo, cambios espontáneos de la señal BOLD que no son atribuidos a un diseño experimental también están presentes \parencite{Fox2007}.
Es así como la resonancia magnética funcional en estado de reposo (\textit{resting-state fMRI}) ha emergido recientemente como una poderosa herramienta que permite medir la conectividad funcional \parencite{Biswal2010}.\par
La resonancia funcional durante el reposo revela fluctuaciones espontáneas de gran amplitud y baja frecuencia (\deactivatequoting\SI{< 0.1}{\hertz}\activatequoting) que pueden ser temporalmente correlacionadas entre áreas relacionadas funcionalmente.
Un único escaneo (de al menos 5 minutos) puede ser usado para estudiar una multitud de circuitos funcionales simultáneamente \parencite{Biswal2010}.

\subsection{Resonancia magnética funcional en adicción}
Las investigaciones de neuroimagen sobre la neurobiología de las adicciones han sido en su mayoría realizadas por técnicas como la tomografía por emisión de positrones (PET) o la fMRI por medio de tareas, donde muchas de estas tareas tienen que ver con la presentación de señales o impulsos relacionados con la sustancia \parencite{Jasinska2014}.
Como se mencionó en la sección \ref{intox}, es común en muchos de los comportamintos adictivos la reacción ante estímulos relacionados con el consumo, induciendo \textit{craving} como consecuencia de un condicionamiento Pavloviano. Revisando la literatura, \textcite{Suckling2017} encontraron que en los sujetos adictos hay un incremento en la activación en regiones prefrontales y orbitofrontales y que las regiones con una reactividad a estímulos relacionados al consumo convergían a la corteza del cíngulo anterior, amígdala y estriado ventral en sujetos adictos a la nicotina, alcohol y cocaína. \par
El control inhibitorio es la supreción de acciones pre-potentes y la resistencia a interferencia de estímulos externos para emplear comportamientos dirigidos a metas.
En individuos con adicción a cocaína se observó actividad incrementada en las cortezas cingulada y prefrontal, regiones frontales inferiores y cerebelo durante la inhibición de respuesta, independientemente de si la acción fue inhibida con exito o no \parencite{Suckling2017}.
En una tarea \textit{go/no-go}, sujetos adictos a cocaína mostraron mayor cantidad de errores por omisión y comisión que controles, lo que se atribuyó a una hipoactivación de la corteza dorsal anterior del cíngulo en los ensayos \textit{stop} \parencite{Kaufman2003}.
En un segundo estudio se encontró que este déficit inhibitorio en usuarios de cocaína era agravado por una mayor carga de memoria de trabajo, y de nueco, una activación de la corteza dorsal anterior del cíngulo fue relacionada con el déficit conductual \parencite{Hester2004}.
Otro estudio que investigaba cómo los sujetos con adicción a cocaína y controles respondían ante recompenzas monetarias por un correcto desempeño en una tarea de atención encontró que aquellos sujetos con adicción a cocaína mostraban señales BOLD reducidas en la corteza orbitofrontal izquierda en ganancias altas comparados con controles además de ser menos sensibles a la diferencia del valor de las recompensas en la actividad de la corteza orbitofrontal y la corteza prefrontal dorsolateral \parencite{Goldstein2007}. \par
\textcite{Connolly2012} exploraron los diferentes patrones de control cognitivo en sujetos adictos a cocaína con diferente tiempo de abstinencia y controles;
entre sus hallazgos encontraron que todos los grupos mostraban un nivel de desempeño similar, los sujetos adictos tenían una mayor activación asociada con el control inhibitorio y el monitoreo de desempeño. Pero más importante, los dos grupos de sujetos adictos mostraron diferencias entre sí en los niveles de activación, sugiriendo diferentes demandas de control cognitivo relacionadas a la duración de abstinencia: el grupo de abstinencia corta tenía una activación en regiones dorsales de los giros frontales medio y superior, mientras que los de abstinencia mayor tendían a reclutar áreas más inferiores, como el giro frontal inferior bilateral.
De igual forma, ambos grupos de adictos presentaron regiones de actividad cerebelares en contraste con los controles, lo que podría sugerir que como usuarios activos de la sustancia tienden a depender de estas regiones como una compensación de la atrofia prefrontal, lo que se mantiene en la abstinencia. \par
Diversos estudios que incorporan el procesamiento emocional con tareas cognitivas indican que la corteza prefrontal dorsolateral es principalmente hiperactiva durante el procesamiento de emociones en individuos con adicción comparado con controles; especialmente en emociones negativas; la corteza cingular anterior ha mostrado resultados mixtos \textemdash{}con más estudios mostrando hipoactividad que hiperactividad\footnote{Es posible que la hiperactividad de la corteza prefrontal dorsolateral esté compensando la hipactividad de la corteza cingulada anterior} \parencite{Goldstein2012a}.\par
Tareas que involucran respuestas emocionales activan el sistema límbico; En un estudio que buscaba observar las diferencias en el procesamiento emocional entre sujetos adictos y controles encontró que había una diferencia inter-grupa significativa en las regiones de procesamiento emocional \parencite{Asensio2010}; el grupo de adicción a cocaína mostró una menor activación en el estriado derecho, tálamo izquierdo, corteza prefrontal dorsolateral y giro parietal con las imágenes placenteras y una hiperactivación del giro parietal superior y corteza prefrontal dorsomedial ante las imágenes placenteras, mostrando una disregulación de valencia en los mecanismos de procesamiento de emociones.
La hipoactivación de la corteza dorsomedial placenteros sugiere una dañada evaluación ante recompensas y disminuida atribución de saliencia y motivacional hacia estímulos placenteros, mientras que la hipoactivación estriatal y dorsolateral prefrontal puede ser la causante de la habilidad reducida de experimentar placer de los estímulos reforzadores naturales (ver sección \ref{abst}).

\subsection{Resonancia magnética funcional en reposo en adicción}
Actualmente son pocos los estudios de conectividad funcional en estado de reposo en el campo de las adicciones, especialmente comparados con los que utilizan las técnicas anteriores.
Relacionado al abuso de la heroína, se ha encontrado alteraciones en la conectividad funcional entre regiones límbicas \textemdash{}como el núcleo accumbens, amígdala, núcleo caudado\textemdash{} y regiones frontales \textemdash{}como la corteza orbitofrontal y el cíngulo \parencite{Ma2010,Tianye2015,Wang2010,Zhang2016}.\par
Son especialmente pocas las investigaciones que han reportado la conectividad funcional de reposo en la adicción a la cocaína.
En estos pacientes se ha observado una disminución en la conectividad del sistema meso-cortico-límbico; entre el área tegmental ventral y el núcleo accumbens, y el tálamo; entre la amígdala y la corteza prefrontal medial; así como entre el hipotálamo y la corteza prefrontal medio-dorsal.
Esta disminución en conectividad estaba negativamente relacionada con el tiempo de adicción \parencite{Gu2010}.
De igual forma, se ha encontrado una correlación negativa entre el \textit{craving} subjetivo y la actividad del giro medial-posterior del cíngulo en adictos a la cocaína.
En el mismo estudio se observó una relación entre las áreas que procesan las señales relacionadas a la droga (corteza orbitofrontal y estriado ventral);
así como una conectividad negativa entre estas áreas y el giro medial posterior del cíngulo \parencite{Wilcox2011}.\par
Se han reportado diferencias interhemisféricas en las regiones frontales entre consumidores y sujetos control;
así como una reducción en la conectividad funcional interhemisférica entre nodos de la red de atención dorsal (áreas latero-frontales bilaterales, premotoras mediales y parietales posteriores), lo que podría sugerir que estas anormalidades se relacionan a los problemas de atención presentados comúnmente en la adicción \parencite{Kelly2011a}.
\textcite{Verdejo-Garcia2014} hallaron menor conectividad funcional entre corteza anterior del cíngulo, tálamo, ínsula y tallo cerebral; así como alteraciones funcionales en los sistemas fronto-límbicos.\par
\textcite{Hu2015} encontraron una conectividad aumentada en los circuitos fronto-estriatales de los usuarios de cocaína, los cuales también presentaron una conexión reducida entre el estriado y las regiones del cíngulo, estriado, hipocampo/amígdala e ínsula. El uso compulsivo de la cocaína fue asociado con un balance entre un aumento de la conectividad anteroestriatal-prefrontal/orbital y una disminución de la conectividad estrato-dorsal anterior cingulada.\par
Los primeros estudios de conectividad efectiva en usuarios de cocaína en abstinencia encontraron una mayor conectividad del área tegmental ventral a: núcleo accumbens, hipocampo y corteza frontal-medial; así como una menor conectividad de la corteza frontal-medial a el área tegmental ventral y del núcleo accumbens a corteza frontal-medial.
A las 72 horas de abstinencia, los fumadores de cocaína presentaron conexiones causales (dirigidas) hacia regiones límbicas, mediales y frontales del sistema MCL que no se presentaron en los controles \parencite{Ray2017,Ray2016}.\par
Estos estudios de neuroimagen y conectividad permiten tener una visión de los circuitos involucrados en la adicción, así como un marco de referencia para evaluar el cambio producido por el tratamiento.

% Resumir los circuitos mencionados? Podría ser intro para las redes seleccionadas Naaa
\section{Teoría de Grafos}
Desde el siglo XIX es bien sabido que los elementos neuronales del cerebro constituyen una formidablemente complicada red estructural. Desde los años 90s, el aumento de nuestro entendimiento de la física de los sistemas complejos ha llevado al desarrollo de la ciencia de las redes, un esfuerzo interdisciplinario de caracterizar la estructura y función de las redes (Bullmore & Sporns).

Las redes estructurales o funcionales del cerebro pueden ser estudiadas por medio de la teoría de grafos siguiendo los siguientes pasos:
\begin{enumerate}
    \item Definir los nodos de la red. Estos pueden ser definidos como los electrodos en un estudio de electroencefalograma o regiones anatómicamente definidas de datos de MRI.
    \item Estimar una medida contínua de asociación entre los nodos. Un ejemplo sería el coeficiente de correlación entre las fluctuaciones de la señal BOLD en una corrida de fMRI.
    \item Generar una matriz de asociación compilando todas las asociaciones entre pares de los nodos.
    \item Calcular los parámetros de redes de interés en este modelo gráfico de red cerebral y compararlos entre grupos o contra parámetros equivalentes en una población de redes aleatorias.
\end{enumerate}

Una red es definida en teoría de grafos como un conjunto de nodos o vértices y \textit{edges} o aristas entre estos. La topología del grafo puede ser cuantitativamente descrita por una amplia variedad de métricas, no obstante, aun no es definido cuáles son las métricas más apropiadas para el análisis de redes cerebrales (Bullmore & Sporns).

Grado de conectividad, distribución de grado y asortatividad; El número de conexiones que enlazan un nodo con el resto de la red es definido como el grado \textemdash{}esta es la medida de red más fundamental y de la cuál últimamente el resto de las medidas se derivan. Los grados de todos los nodos de la red forman la distribución de grado. En redes aleatorias todas las conexiones son igualmente probables, resultando en una distribución Gaussiana y simétricamente centrada. Las redes complejas, como las redes de conectividad funcional, tienden a tener distribuciones de grado no-Gaussianas, frecuentemente con una cola alargada hacia altos grados de conectividad. Bullmore & sporns.

Coeficiente de agrupamiento (o \textit{clustering}); Si los nodos vecinos más cercanos a un nodo están también conectados entre sí, los tres forman un \textit{cluster}. El coeficiente de agrupamiento cuantifica este número de conexiones que existen entre los vecinos más cercanos de un nodo como una proporción del máximo número posible de conexiones.

Largo de camino y eficiencia; El largo de camino es el número más corto de aristas que deben ser recorridas para ir de un nodo a otro. La eficiencia es inversamente relativa al largo de camino pero es numéricamente más sencilla de usar para estimar distancia topológicas entre elementos de grafos desconectados.

Densidad de conexión o costo; La densidad de conexión es el numero actual de aristas en el grafo como una proporción del número total de posibles conexiones y es el estimador más simple del costo físico \textemdash{}por ejemplo el requerimento de energía o cualquier otro recurso físico\textemdash{}de una red.

Los hubs son nodos con alto grado de conectividad o alta centralidad. La centralidad de un nodo mide cuántos largos de camino más cortos entre los nodos de la red pasan a través de él. Un nodo con alto grado de centralidad es entonces crucial para una comunicación eficiente.

Muchos trastornos neurológicos y psiquiátricos pueden ser descritos como síndromes de disconectividad; la emergencia de síntomas o deficiencia funcional de estos trastornos puede estar relacionada con la disrupción o integración anormal de regiones cerebrales spacialmente distribuidas que normalmente constituirían una red a gran escala. Una aplicación de la teoría de redes complejas, en este caso, sería el proveer nuevas medidas para cuantificar las diferencias entre grupos de pacientes y grupos de comparación apropiados (Bullmore & Sporns).











La topología de pequeño mundo de las redes funcionales del cerebro se preserva entre las múltiples bandas de frecuencia y tareas conductuales (Sporns & Honey).

La combinación de \textit{clustering} local e interacción global proveen un sustrato estructural para la coexistencia de segregación e integración funcional en el cerebro, un sello distintivo de la complejidad de la red cerebral (Tononti y sporns & tononi) (Sporns y honey). La necesidad de la red para satisfacer simultaneamente las demandas opuestas del procesamiento global y local, de la misma forma que tienden a minimizar el número de pasos necesarios de procesamiento (Sporns & Honey). Las redes cerebrales preservan las caracteristicas topologicas globales (manteniendo continuamente el balance de eficiencia global y local) al mismo tiempo que adaptan flexiblemente la topología necesaria para satisfacer las demandas de la tarea cambiante.






















La teoría de grafos es un nuevo enfoque que ha venido tomando fuerza en el campo de la neuroimagen como una forma de describir comprensivamente la red de elementos y conexiones que forman el cerebro humano.
Un grafo es una representación de una red. Consiste en un conjunto de vértices o nodos y un conjunto de \textit{edges} o aristas. La presencia de una arista entre dos vértices indica la presencia de algún tipo de interacción o conexión \parencite{Stam2007}.
Tanto redes estructurales como funcionales pueden ser exploradas usando teoría de grafos; las métricas de la red pueden ser computadas para extraer las características de la topología del grafo.
Esta topología se presta para ser comparada entre sujetos o grupos de investigación \parencite{Bullmore2009a,Sporns2011}.\par
Se ha apreciado que muchos trastornos neurológicos y psiquiátricos pueden ser descritos como síndromes de disconectividad.
Una aplicación de la teoría de grafos en este contexto puede ser el de proveer nuevas medidas para cuantificar diferencias entre grupos de pacientes y grupos apropiados de comparación \parencite{Bullmore2009a}.

\subsubsection{Medidas topográficas}
Buscar formulas de métricas y su referencia
La topología de una red o grafo puede describirse por medio de diversas métricas. Las principales medidas para describir las redes cerebrales pueden clasificarse en las siguientes categorías.\par
\theoremstyle{definition}
\newtheorem*{def1}{Costo}
\newtheorem*{def2}{Segregación}
\newtheorem*{def3}{Integración}
\newtheorem*{def4}{Pequeño Mundo}
El número de conexiones de cada nodo(grado de conectividad). El grado medio de todos los nodos refleja la densidad de la red.
