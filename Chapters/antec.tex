\section{Adicción a Cocaína}

La adicción es una consecuencia patológica del aprendizaje sobre recompensas, donde se alteran las rutas meso-cortico-límbicas de la dopamina \parencite{Volkow2016}.
El término de adicción es usado para indicar la etapa más severa y crónica del trastorno por consumo de sustancias, donde hay una pérdida substancial de autocontrol \parencite{Volkow2016}.
En México se ha reportado un aumento en la prevalencia del 1.8\% con respecto al consumo de cualquier droga para la población de 12 a 65 años, agregándose 100,000 personas como dependientes (de 450,000 el 2008 a 550,000 en el 2011)\parencite{InstitutoNacionaldePsiquiatriaRamondelaFuenteMuniz2012a}.
La adicción es entonces un trastorno recurrente que puede caracterizarse por:
\begin{enumerate*}
    \item{compulsión en buscar y consumir la sustancia; }
    \item{pérdida de control limitando el consumo, y; }
    \item{emergencia de un estado emocional negativo reflejando un síndrome motivacional de abstinencia}
\end{enumerate*}.
Por lo que puede dividirse en tres estados: intoxicación, abstinancia y afecto negativo, y, \textit{craving} \parencite{Koob2010a}.\par
El \textit{craving}, o el deseo fuerte e intenso de consumir la sustancia, tanto para volver a sentir los efectos eufóricos como para evitar los aspectos de abstinencia provocados por su ausencia, es un elemento clave en la recaída \parencite{Koob2010a}. Existe una baja regulación de señalización de dopamina (DA) en regiones prefrontales del cerebro y sus circuitos asociados, afectando los procesos ejecutivos \parencite{Goldstein2012a}. Esto crea un desbalance que es crucial tanto para el desarrollo gradual del comportamiento compulsivo como para la asociada inhabilidad a resistirse a las fuertes ansias por consumir o para seguir las decisiones de parar tomando la droga \parencite{Volkow2016}. El \textit{craving} es entonces, por estas características, un punto clave como enfoque para el tratamiento de las adicciones.

\section{Tratamiento para la Adicción a la Cocaína}

