\section{Adicción a Cocaína}
% Buscar mas sobre bases biológicas de Adicción
% Agregar impulsividad
La adicción es una consecuencia patológica del aprendizaje sobre recompensas, donde se alteran las rutas meso-córtico-límbicas de la dopamina \parencite{Volkow2016}.
El término de adicción es usado para indicar la etapa más severa y crónica del trastorno por consumo de sustancias, donde hay una pérdida substancial de auto-control \parencite{Volkow2016}.
En México se ha reportado un aumento en la prevalencia del 1.8\% con respecto al consumo de cualquier droga para la población de 12 a 65 años, agregándose 100,000 personas como dependientes (de 450,000 el 2008 a 550,000 en el 2011)\parencite{InstitutoNacionaldePsiquiatriaRamondelaFuenteMuniz2012a}.
La adicción es entonces un trastorno recurrente que puede caracterizarse por:
\begin{enumerate*}
    \item{compulsión en buscar y consumir la sustancia; }
    \item{pérdida de control limitando el consumo, y; }
    \item{emergencia de un estado emocional negativo reflejando un síndrome motivacional de abstinencia}
\end{enumerate*}.
Por lo que puede dividirse en tres estados: intoxicación, abstinencia y afecto negativo, y, \textit{craving} \parencite{Koob2010a}.\par
El \textit{craving}, o el deseo fuerte e intenso de consumir la sustancia, tanto para volver a sentir los efectos eufóricos como para evitar los aspectos de abstinencia provocados por su ausencia, es un elemento clave en la recaída \parencite{Koob2010a}.
Existe una baja regulación de señalización de dopamina (DA) en regiones prefrontales del cerebro y sus circuitos asociados, afectando los procesos ejecutivos \parencite{Goldstein2012a}.
Esto crea un desbalance que es crucial tanto para el desarrollo gradual del comportamiento compulsivo como para la asociada inhabilidad a resistirse a las fuertes ansias por consumir o para seguir las decisiones de parar tomando la droga \parencite{Volkow2016}.
El \textit{craving} es entonces, por estas características, un punto clave como enfoque para el tratamiento de las adicciones.

\section{Tratamiento para la Adicción a la Cocaína}
Actualmente no existe una cura para la adicción.
El tratamiento consiste en un abordaje multidisciplinario que ayuda al manejo de la enfermedad.
Generalmente este consiste en el apoyo psicoterapéutico, consejería y tratamiento farmacológico en conjunto. \par
No obstante, la naturaleza crónica de la adicción hace que la recaída no sea solo posible, sino probable \parencite{NIDA.}.
Estudios de seguimiento a 1 año postratamiento han encontrado que solo del 40-60\% se mantienen en abstinencia \parencite{McLellan1980} \footnote{Datos de tratamientos en EEUU}.\par
Debido a esto es de suma importancia la revisión del manejo actual de la adicción y la búsqueda de nuevos tratamientos alternativos que sean más efectivos.

\subsection{Tratamiento farmacológico}
No existe actualmente un tratamiento farmacológico aprobado por la FDA (Administración de comida y drogas, por sus siglas en inglés) de Estados Unidos para la dependencia a la cocaína.
Los medicamentos utilizados comúnmente son los mismos que aquellos usados para tratar la epilepsia o los espasmos musculares, principalmente con el fin de aliviar la ansiedad y la agitación resultantes de la adicción a la cocaína.
Algunos de estos son: gabapentina, un anticonvulsivante análogo a GABA; modafinilo, promueve el estado de alerta inhibiendo la recaptura de dopamina; topiramato, anticonvulsivante que alivia la agitación; vigabatrina, anti-epiléptico usado para el \textit{craving}inhibiendo el catabolismo de GABA; y, baclofeno, relajante muscular agonista de GABA \parencite{Volkow2007b}.\par
De todos los medicamentos usados para tratar la adicción a la cocaína, disulfiram es el que ha tenido más resultados exitosos consistentes \parencite{Volkow2007b}.
Usualmente utilizado como tratamiento a la adicción al alcohol por medio de la inducción de una reacción adversa al alcohol, disulfiram también a sido prescrito para desalentar el uso de la cocaína.
Aún no se conoce específicamente como funciona, pero sus efectos pueden estar relacionados a su capacidad de inhibir una enzima que convierte a la dopamina en noradrenalina.
Sin embargo, no es efectivo para todas las personas, ya que se ha visto que ciertas variaciones genéticas influyen en la efectividad del tratamiento \parencite{Gaval-Cruz2009a, Volkow2007b}.\par

\subsection{Tratamiento comportamental}
Junto al tratamiento farmacológico, usualmente se lleva a cabo un abordaje comportamental por medio de psicoterapia y consejería.
Entre los modelos más utilizados dentro de estas áreas están:
la psicoterapia cognitivo-conductual, donde el paciente aprende a identificar y corregir comportamientos problemáticos aplicando una serie de técnicas que pueden ser usadas para detener el abuso de la sustancia y abordar los problemas que suelen ocurrir en conjunto;
el manejo de contingencias, donde por medio de recompensas tangibles se refuerzan los comportamientos positivos como la abstinencia;
intervención motivacional, es un abordaje de consejería donde se ayuda al paciente a resolver su ambivalencia sobre el compromiso al tratamiento y parar el consumo de la sustancia;
y la terapia familiar conductual, que intenta abordar no solo el consumo de la droga, sino también los problemas en conjunto, como trastornos de conducta, maltrato infantil, depresión, conflictos familiares y desempleo \parencite{Volkow2008}.

\section{Estimulación Magnética Transcraneal Repetitiva}
% Buscar estudios de Hanlon
% Agregar subsección de TMS como tratamiento de adicción???
A pesar de ser originalmente desarrollada como una herramienta diagnóstica, la Estimulación Magnética Transcraneal (EMT) puede modular, transitoria ---o duraderamente--- la excitabilidad cortical por medio de la aplicación de pulsos electromagnéticos localizados\parencite{Horvath2011a}.\par
La EMT repetitiva (EMTr) es una variación de la EMT donde la estimulación es provista en varias sesiones con las mismas condiciones para crear excitación o inhibición a largo plazo en la corteza cerebral.
En EMTr las actividades eléctricas en el cerebro son influenciadas por los campos magnéticos. La inducción electromagnética generada por EMTr es indolora y pasa sin daño por la piel y el cráneo \parencite{Noohi2016}.
Se ha demostrado que la EMTr puede perturbar la actividad neuronal por períodos que exceden la duración de la estimulación; por lo tanto, es una técnica de neuromodulación que produce cambios plásticos \parencite{Horvath2011a}.\par
EMTr sobre la Corteza Prefrontal (CPF) ha mostrado tener efectos moduladores en los sistemas dopaminérgicos mesolímbicos y mesoestriatales.
Se encontró que la Emtr de alta frecuencia sobre la CPF inducía liberación subcortical de dopamina (DA) en el núcleo caudado\parencite{Strafella2001}.
La EMTr sobre la Corteza Prefrontal Dorsolateral (DLPFC) izquierda modula la liberación de DA en: Corteza Anterior Cingulada (CAC) y Corteza Orbitofrontal (COF) en el mismo hemisferio que la estimulación \parencite{Cho2009}.
Sesiones repetidas de EMTr sobre la CPF son sugeridas para reducir el \textit{craving}, búsqueda de drogas y, eventualmente, el consumo y recaída\parencite{Amiaz2009}.\par
\textcite{Bellamoli2014a} realizaron una revisión de la literatura buscando protocolos de EMTr sobre el \textit{craving} y consumo de nicotina, alcohol y cocaína. Basándose en la evidencia encontrada, concluyeron que la EMTr puede ser clasificada como probablemente efectiva en el tratamiento de la adicción.
Las características neuromoduladoras y los cambios plásticos observados como efecto de la EMTr hacen de esta un tratamiento efectivo factible para alterar los circuitos involucrados en el \textit{craving} de la adicción, lo que, por efecto, llevaría a la disminución de la recaída y una mayor efectividad de tratamiento.

\section{Conectividad Funcional en la Adicción}
% Separar Bases de FMRI y FMRI en adicción como subsección
% Agregar bases de resonancia magnética
Mucho de lo que actualmente se conoce sobre función cerebral ha venido de estudios donde se miden los cambios en actividad neuronal y conducta después de la administración de una tarea o estímulo (\textit{task-based fMRI}), sin embargo, cambios espontáneos de la señal BOLD que no son atribuidos a un diseño experimental también están presentes \parencite{Fox2007}.
Es así como la resonancia magnética funcional en estado de reposo (\textit{resting-state fMRI} o RS-RMf) ha emergido recientemente como una poderosa herramienta que permite medir la conectividad funcional \parencite{Biswal2010}\par
La resonancia funcional durante el reposo revela fluctuaciones espontáneas de gran amplitud y baja frecuencia ($<0.1 Hz$) que pueden ser temporalmente correlacionadas entre áreas relacionadas funcionalmente.
Un único escaneo (de al menos 5 minutos) puede ser usado para estudiar una multitud de circuitos funcionales simultáneamente \parencite{Biswal2010}.\par
% Aquí empieza neuroimagen y adicciones
Las investigaciones de neuroimagen sobre la neurobiología de las adicciones han sido en su mayoría realizadas por técnicas como la tomografía por emisión de positrones (PET) o la RMf por medio de tareas, donde muchas de estas tareas tienen que ver con la presentación de señales o impulsos relacionados con la sustancia \parencite{Jasinska2014}. Actualmente son pocos los estudios de conectividad funcional en estado de reposo en el campo de las adicciones, especialmente comparados con los que utilizan las técnicas anteriores. \par
Relacionado al abuso de la heroína, se ha encontrado alteraciones en la conectividad funcional entre regiones límbicas ---como el Núcleo Accumbens (NA), amígdala, Núcleo Caudado (NC)--- y regiones frontales ---como la COF y el cíngulo\parencite{Ma2010,Tianye2015,Wang2010,Zhang2016}.\par
Son especialmente pocas las investigaciones que han reportado la conectividad funcional de reposo en la adicción a la cocaína.
En estos pacientes se ha observado una disminución en la conectividad del sistema meso-cortico-límbico (MCL); entre el Área Tegmental Ventral (ATV) y el NA, y el Tálamo; entre la Amígdala y la CPF medial; así como entre el Hipotálamo y la CPF medio-dorsal.
Esta disminución en conectividad estaba negativamente relacionada con el tiempo de adicción \parencite{Gu2010}.
De igual forma, se ha encontrado una correlación negativa entre el \textit{craving} subjetivo y la actividad del giro medial-posterior del cíngulo en adictos a la cocaína.
En el mismo estudio se observó una relación entre las áreas que procesan las señales relacionadas a la droga (COF y estriado ventral);
así como una conectividad negativa entre estas áreas y el giro medial posterior del cíngulo \parencite{Wilcox2011}.\par
Se han reportado diferencias interhemisféricas en las regiones frontales entre consumidores y sujetos control; así como una reducción en la conectividad funcional interhemisférica entre nodos de la red de atención dorsal (áreas latero-frontales bilaterales, premotoras mediales y parietales posteriores), lo que podría sugerir que estas anormalidades se relacionan a los problemas de atención presentados comúnmente en la adicción \parencite{Kelly2011a}.
\textcite{Verdejo-Garcia2014} hallaron menor conectividad funcional entre CAC, tálamo, ínsula y tallo cerebral; así como alteraciones funcionales en los sistemas fronto-límbicos.\par
\textcite{Hu2015} encontraron una conectividad aumentada en los circuitos fronto-estriatales de los usuarios de cocaína, los cuales también presentaron una conexión reducida entre el estriado y las regiones del cíngulo, estriado, hipocampo/amígdala e ínsula. El uso compulsivo de la cocaína fue asociado con un balance entre un aumento de la conectividad anteroestriatal-prefrontal/orbital y una disminución de la conectividad estrato-dorsal anterior cingulada.\par
Los primeros estudios de conectividad efectiva en usuarios de cocaína en abstinencia encontraron una mayor conectividad del ATV a: NA, hipocampo y CFM; así como una menor conectividad de la CFM a ATV y del NA a CFM.
A las 72 horas de abstinencia, los fumadores de cocaína presentaron conexiones causales (dirigidas) hacia regiones límbicas, mediales y frontales del sistema MCL que no se presentaron en los controles \parencite{Ray2017,Ray2016}.\par
Estos estudios de neuroimagen y conectividad permiten tener una visión de los circuitos involucrados en la adicción, así como un marco de referencia para evaluar el cambio producido por el tratamiento.
% Resumir los circuitos mencionados? Podría ser intro para las redes seleccionadas
\subsection{Teoría de Grafos}
La teoría de grafos es un nuevo enfoque que ha venido tomando fuerza en el campo de la neuroimagen como una forma de describir comprensivamente la red de elementos y conexiones que forman el cerebro humano.
Un grafo es una representación de una red. Consiste en un conjunto de vértices o nodos y un conjunto de \textit{edges} o aristas. La presencia de una arista entre dos vértices indica la presencia de algún tipo de interacción o conexión \parencite{Stam2007}.
Tanto redes estructurales como funcionales pueden ser exploradas usando teoría de grafos; las métricas de la red pueden ser computadas para extraer las características de la topología del grafo.
Esta topología se presta para ser comparada entre sujetos o grupos de investigación \parencite{Bullmore2009a,Sporns2011}.\par
Se ha apreciado que muchos trastornos neurológicos y psiquiátricos pueden ser descritos como síndromes de disconectividad.
Una aplicación de la teoría de grafos en este contexto puede ser el de proveer nuevas medidas para cuantificar diferencias entre grupos de pacientes y grupos apropiados de comparación \parencite{Bullmore2009a}.

\subsubsection{Medidas topográficas}
Buscar formulas de métricas y su referencia
La topología de una red o grafo puede describirse por medio de diversas métricas. Las principales medidas para describir las redes cerebrales pueden clasificarse en las siguientes categorías.\par
\theoremstyle{definition}
\newtheorem*{def1}{Costo}
\newtheorem*{def2}{Segregación}
\newtheorem*{def3}{Integración}
\newtheorem*{def4}{Pequeño Mundo}
El número de conexiones de cada nodo(grado de conectividad). El grado medio de todos los nodos refleja la densidad de la red.
