
Diversos estudios han demostrado como los patrones de activación BOLD son capaces de predecir la recaída en poblaciones con adicción, así como predecir respuestas a tratamiento \parencite{Suckling2017}.

% Mexico prevalence add to justification instead
En México se ha reportado un aumento significativo en el consumo de drogas ilegales en los últimos años pasando del 7.2\% en el 2011, al 9.9\% de la población total en el 2016.
Por su parte, la dependencia a drogas en el último año fue reportada por un 0.6\% de la población (1.1\% de hombres y 0.2\% de mujeres).
De las drogas ilícitas, la cocaína ocupa el segundo mayor lugar en su consumo, después de la mariguana \parencite{Villatoro-Velazques2017}.

En ocasiones, no se encuentra diferencia entre sujetos sanos y adictos en índices de valencia o agitación \textemdash{}incluso en reacciones autonómicas\textemdash{} frente a estímulos relacionados con el consumo de drogas \parencite{Goldstein2012a}, lo que sugiere que las medidas de neuroimagen son más sensibles en detectas diferencias de grupo.
