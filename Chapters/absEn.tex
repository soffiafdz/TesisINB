\selectlanguage{english}
\begin{center}
    \large\textbf{Brain Connectivity in Cocaine-addicted Patients after Repetitive Transcranial Magnetic Stimulation Treatment}
\end{center}
\begin{quotation}
    \noindent
There is still few and inadequate research being done on the effectivity of repetitive transcranial magnetic stimulation (rTMS) as a treatment for cocaine addiction and even fewer using graph theory analysis methods in the study of addiction. In this longitudinal, monocentric, double-blind placebo controlled clinical trial, we explored the topology differences between cocaine-dependent participants and healthy controls and used that data to assess the clinical and topological changes after a rTMS treatment over the left dorsolateral prefrontal cortex. The global topology and clinical metrics were measured in 40 cocaine-dependent, treatment seeking participants before and after receiving two daily sessions of either real or sham rTMS for two weeks. We also explored these same measures after receiving weekly maintenance sessions for three ($n=16$) and  six months ($n=11$). Our preliminary analysis showed significant difference in the cost, efficiency and small worldness of the networks between our sample of cocaine-dependent participants and healthy controls. Mixed effects models showed a significant interaction of stimulation group and stage of treatment for both craving and impulsivity. There were significant changes in the cost of the networks and the small worldness attributable to the rTMS treatment. All of these changes were maintained after the three months of maintenance sessions and showed slight decay by the sixth month. These results provide evidence for the efficacy of rTMS as a promising alternative treatment for addiction as well as the appropriateness of graph theory analyses methods for the exploration of the nature, evolution and treatment of addiction.
\end{quotation}
\clearpage
\selectlanguage{spanish}
