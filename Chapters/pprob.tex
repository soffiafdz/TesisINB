\section{Justificación}
La adicción a sustancias es un importante problema de salud pública en México y en el mundo.
En México se ha reportado un aumento significativo en el consumo de drogas ilícitas en los últimos años, aumentando del 7.2\% en el 2011 al 9.9\% de la población total en el 2016.
La dependencia a drogas\footnote{Reportada en el último año} fue reportada por un 0.6\% de la población \textemdash{}1.1\% de hombres y 0.2\% de mujeres en el 2016.
De estas drogas ilícitas, la cocaína ocupa el segundo mayor lugar en su consumo, después de la mariguana \parencite{Villatoro-Velazques2017}\par
Aunque la mariguana sea la droga de mayor consumo, el presente proyecto se enfoca en la dependencia a la cocaína debido a su mayor efecto adictivo\footnote{Esto se observa en un mayor y más intenso \textit{craving}} e impacto sobre la salud y funcionalidad social de sus usuarios, tanto a corto como largo plazo. \par
El tratamiento actual para la adicción, específicamente para la dependencia a cocaína, es insuficiente. El enfoque de tratamiento actual es enfocado principal Con una alta tasa de recaída \parencite{McLellan2010a}





Diversos estudios han demostrado como los patrones de activación BOLD son capaces de predecir la recaída en poblaciones con adicción, así como predecir respuestas a tratamiento \parencite{Suckling2017}.

En ocasiones, no se encuentra diferencia entre sujetos sanos y adictos en índices de valencia o agitación \textemdash{}incluso en reacciones autonómicas\textemdash{} frente a estímulos relacionados con el consumo de drogas \parencite{Goldstein2012a}, lo que sugiere que las medidas de neuroimagen son más sensibles para detectar diferencias de grupo


La corteza prefrontal está densamente interconectada con otras regiones cerebrales corticales y subcorticales, incluyendo redes como la DMN y la red de atención dorsal (implicadas en procesos de control ejecutivos como la atención e inhibición\parencite{Goldstein2012a}.

Mediación frontal de un circuito neural involucrado en la respuesta al craving puede funcional como un objetivo para intervenciones cognitivas \textit{top-down}que pueden tener un beneficio terapéutico. Intervenciones que refuercen a un debilitado pero aun funcional circuito fronto-accumbal puede incrementar la habilidad de usuarios a cocaína para bloquear o reducir la respuesta al craving \parencite{Volkow2010a}.
