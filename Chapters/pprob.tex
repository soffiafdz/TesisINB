\section{Justificación}
La adicción a sustancias es un importante problema de salud pública en México y en el mundo.
En México se ha reportado un aumento significativo en el consumo de drogas ilícitas en los últimos años, aumentando del 7.2\% en el 2011 al 9.9\% de la población total en el 2016.
La dependencia a drogas\footnote{Reportada en el último año} fue reportada por un 0.6\% de la población \textemdash{}1.1\% de hombres y 0.2\% de mujeres en el 2016.
De estas drogas ilícitas, la cocaína ocupa el segundo mayor lugar en su consumo, después de la mariguana \parencite{Villatoro-Velazques2017}\par
Aunque la mariguana sea la droga de mayor consumo, el presente proyecto se enfoca en la dependencia a la cocaína debido a su mayor efecto adictivo\footnote{Esto se observa en un mayor y más intenso \textit{craving}} e impacto sobre la salud y funcionalidad social de sus usuarios, tanto a corto como largo plazo. \par
El tratamiento actual para la adicción, específicamente para la dependencia a cocaína, es insuficiente.
El campo de la psiquiatría principalmente se apoya de listas de síntomas y marcadores de consumo.
Hasta el momento no hay biomarcadores clínicos útiles para la adicción a sustancias.
Un pobre entendimiento del cerebro humano adicto y de los efectos complejos de las drogras en los distintos mecanismos neurobiológicos y circuitos neuronales son las razones principales de la falla de desarrollar tratamientos efectivos, donde se presenta una tasa de recaída cercana al 50\% \parencite{McLellan2000a}. \par
Una nueva propuesta de tratamiento que pretende aprovechar las investigaciones recientes en los neurocircuitos de la adicción es el de la EMTr.
Con base en la involucración de la mediacion frontal sobre la respuesta al craving, esta zona puede funcionar como un objetivo terapéutico. Intervenciones como la EMTr que refuerzan a un debilitado pero aún funcional circuito fronto-accumbal pueden incrementar la habilidad de usuarios a cocaína para bloquear o reducir la respuesta al \textit{craving} \parencite{Volkow2010a}. \par
Varios investigadores han buscado explorar la eficacia de este tratamiento con resultados prometedores en la reducción clínica del craving \parencite{Rachid2018}.
No obstante, los estudios son aún insuficientes; muy pocos de los estudios clínicos cuentan con un grupo control de comparación y estos son diseños ciegos sencillos \parencite{Kearney-Ramos2018a, Kearney-Ramos2019, Terraneo2016,Hanlon2015}.
El único estudio doble-ciego contaba con una muestra de solo 10 sujetos \parencite{Bolloni2016}.
Y, aunque se ha demostrado que las medidas de neuroimagen son más sensibles para detectar diferencias de grupo en valencia o agitación ante estímulos \parencite{Goldstein2012a} y capaces de predecir la recaída y respuesta a tratamiento \parencite{Suckling2017}, solamente tres de los estudios clínicos de EMTr exploraban medidas cerebrales con neuroimagen \parencite{Kearney-Ramos2018a, Kearney-Ramos2019, Hanlon2015}. \par
\textcite{Steele2018} argumentan que para diagnosticar y tratar efectivamente a los pacientes adictos a sustancias, en vez de enfocarse en una región cerebral o neurotransmisonr específico, como se ha venido realizando los últimos años, un mejor entendimiento de los efectos de la condición sobre la organización topológica y las redes de conectividad cerebral puede tener una mucho mayor importancia estratégica. \par
Es por eso que en el presente proyecto, se pretende evaluar la efectividad del tratamiento con EMTr en adicción a la cocaína, siguiendo los lineamientos de \textcite{Ekhtiari2019}, en un estudio doble-ciego a largo plazo (3 meses) y explorar los efectos en mejoría clínica y sobre la topología de las redes de conectividad cerebral.

\section{Pregunta de investigación}
¿Existen cambios en la topología de redes de conectividad funcional en pacientes con adicción a la cocaína después de un tratamiento de estimulación magnética transcraneal repetitiva a corto y largo plazo?

\section{Objetivos}
\subsection{General}
\begin{enumerate}[label=General., left= \parindent]
    \item Evaluar los cambios en conectividad cerebral funcional utilizando métodos de teoría de grafos y su posible relación con la mejoría clínica después de un tratamiento con estimulación magnética transcraneal repetitiva en pacientes con adicción a cocaína a corto (2 semanas) y largo plazo (3 meses) y su relación con la topología de red en pacientes sanos.
\end{enumerate}
\subsection{Específicos}
\begin{enumerate}[label=Específico \arabic*., left= \parindent]
    \item Ver si existe una mejoría clínica después del tratamiento de EMTr a corto y largo plazo.
    \item Explorar la topología de la red de conectividad funcional en reposo de la etapa basal y después de dos semanas y tres meses de tratamiento de EMTr.
    \item Comparar la topología de red entre grupos: \begin{enumerate*}[label=\emph{\alph*})]
        \item estimulación real,
        \item estimulación sham, y
        \item controles sanos\end{enumerate*}.
    \item Buscar si existe relación entre los cambios sintomáticos y de topología después del tratamiento.
\end{enumerate}

\section{Hipótesis}
\subsection{Clínicas}
\begin{enumerate}[label=Hipótesis \arabic*., left= \parindent]
    \item Habrá una disminución en el craving atribuible al tratamiento de EMTr a las dos semanas y tres meses.
    \item Habrá una disminución en la medida de impulsividad atribuible al tratamiento de EMTr a las dos semanas y tres meses.
\end{enumerate}
\subsection{Topológicas}
    \subsubsection{Basales}
    \begin{enumerate}[resume,label=Hipótesis \arabic*., left= \parindent]
        \item Los pacientes con adicción presentarán mayor hiperconectividad (\emph{fuerza} y \emph{densidad}) en sus redes de conectividad que los controles sanos.
        \item Las redes de conectividad en pacientes adictos presentarán menores índices de eficiencia (\emph{escalar de pequeño-mundo} y \emph{eficiencia local} y \emph{global}) que las de los controles sanos.
    \end{enumerate}
    \subsubsection{Relacionadas con EMTr}
    \begin{enumerate}[resume,label=Hipótesis \arabic*., left= \parindent]
        \item La conectividad de la red (\emph{fuerza} y \emph{densidad}) disminuirá con el tratamiento real de EMTr en comparación con \textit{sham}.
        \item La eficiencia de la red (\emph{eficiencia local}, \emph{global} y \emph{pequeño-mundo}) aumentará con el tratamiento real de EMTr en comparación del \textit{sham}.
    \end{enumerate}
    \subsubsection{Relacionadas con mejoría clínica}
    \begin{enumerate}[resume,label=Hipótesis \arabic*., left= \parindent]
        \item La disminución en conectividad de red estará asociada a una disminución en \textit{craving}.
        \item El aumento en eficiencia de la red estará asociado a una disminución en \textit{craving}.
        \item La disminución en conectividad de red estará asociada a una disminución en impulsividad.
        \item El aumento en eficiencia de la red estará asociado a una disminución en impulsividad.
    \end{enumerate}

