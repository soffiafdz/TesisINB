\section{Justificación}
La adicción a sustancias es un problema importante social y de salud en México y en el mundo.
En México se ha reportado un aumento significativo en el consumo de drogas ilícitas en los últimos años, aumentando del 7.2\% en el 2011 al 9.9\% de la población total en el 2016.
La dependencia a drogas\footnote{Reportada en el último año.} fue reportada por un 0.6\% de la población \textemdash{}1.1\% de hombres y 0.2\% de mujeres, en el 2016.
De estas drogas ilícitas, la cocaína ocupa el segundo lugar en su consumo, después de la mariguana \parencite{Villatoro-Velazques2017}.\par
Aunque la mariguana sea la droga de mayor consumo, el presente proyecto se enfoca en la dependencia a la cocaína debido a su mayor efecto adictivo\footnote{Esto se observa en un mayor y más intenso \textit{craving}.} e impacto sobre la salud y funcionalidad social de sus usuarios, tanto a corto como largo plazo. \par
El tratamiento actual para la adicción, específicamente para la dependencia a cocaína, es insuficiente.
El campo de la psiquiatría principalmente se apoya de listas de síntomas y marcadores de consumo.
Hasta el momento no hay biomarcadores clínicos útiles para la adicción a sustancias.
Un pobre entendimiento de los efectos de la adicción en el cerebro humano y de los efectos complejos de las drogas en los distintos mecanismos neurobiológicos y circuitos neuronales, son las razones principales de la falla de desarrollar tratamientos efectivos, ya que el tratamiento habitual presenta una tasa de recaída cercana al 50\% \parencite{McLellan2000a}. \par
Una nueva propuesta de tratamiento que pretende aprovechar las investigaciones recientes en los neurocircuitos de la adicción es el de la EMTr.
Con base en la involucración de la mediación frontal sobre la respuesta al \textit{craving} esta zona puede funcionar como un objetivo terapéutico. Intervenciones como la EMTr que refuerzan a un debilitado pero aún funcional circuito fronto-accumbal, pueden incrementar la habilidad de usuarios a cocaína para bloquear o reducir la respuesta al \textit{craving} \parencite{Volkow2010a}. \par
Varios investigadores han buscado la eficacia de este tratamiento con resultados prometedores en la reducción clínica del \textit{craving} \parencite{Rachid2018}.
No obstante, los estudios son aún insuficientes; muy pocos de los estudios clínicos cuentan con un grupo control de comparación y estos son diseños ciegos sencillos \parencite{Kearney-Ramos2018a, Kearney-Ramos2019, Terraneo2016,Hanlon2015}.
El único estudio doble-ciego contaba con una muestra de solo diez sujetos \parencite{Bolloni2016}.
Y, aunque se ha demostrado que las medidas de neuroimagen son más sensibles para detectar diferencias de grupo en valencia o agitación ante estímulos \parencite{Goldstein2012a} y capaces de predecir la recaída y respuesta a tratamiento \parencite{Suckling2017}, solamente tres de los estudios clínicos de EMTr exploraban medidas cerebrales con neuroimagen \parencite{Kearney-Ramos2018a, Kearney-Ramos2019, Hanlon2015}. \par
\textcite{Steele2018} argumentan que para diagnosticar y tratar efectivamente a los pacientes adictos a sustancias, en vez de enfocarse en una región cerebral o neurotransmisor específico, como se ha venido realizando los últimos años, un mejor entendimiento de los efectos de la condición sobre la organización topológica y las redes de conectividad cerebral puede tener una mucho mayor importancia estratégica. \par
Es por eso que en el presente proyecto, se pretende evaluar la efectividad del tratamiento con EMTr en adicción a la cocaína, siguiendo los lineamientos de \textcite{Ekhtiari2019}, en un estudio doble-ciego a largo plazo (dos semanas de tratamiento agudo; tres y seis meses de mantenimiento) y explorar los efectos en mejoría clínica y sobre la topología de las redes de conectividad cerebral.

\section{Pregunta de investigación}
¿Existen cambios en la topología de redes de conectividad funcional en pacientes con adicción a la cocaína después de un tratamiento de estimulación magnética transcraneal repetitiva a corto y largo plazo?

\section{Objetivos}
\subsection{General}
\begin{enumerate}[label=General., left= \parindent]
    \item Evaluar los cambios en conectividad cerebral funcional utilizando métodos de ciencia de redes y su posible relación con la mejoría clínica después de un tratamiento con estimulación magnética transcraneal repetitiva en pacientes con adicción a cocaína (dos semanas) y sesiones de mantenimiento a largo plazo (tres y seis meses) y su relación con la topología de redes en sujetos control.
\end{enumerate}
\subsection{Específicos}
\begin{enumerate}[label=Específico \arabic*., left= \parindent]

    \item Comparar la topología de redes entre los pacientes diagnosticados con adicción a la cocaína y un grupo control.
    \item Definir una posible mejoría clínica después de dos sesiones de tratamiento de EMTr
    \item Observar si la mejoría persiste con sesiones de mantenimiento semanales de EMTr (tres y seis meses).
    \item Observar si existen cambios en la topología de redes después del tratamiento y sesiones de mantenimiento de EMTr.
    \item Buscar si existe una relación entre los cambios sintomáticos y de topología después del tratamiento y mantenimiento de EMTr.
\end{enumerate}

\section{Hipótesis}
\subsection{Clínicas}
\begin{enumerate}[label=Hipótesis \arabic*., left= \parindent]
    \item Habrá una mayor disminución en el \textit{craving} a las dos semanas, tres y seis meses en comparación con la medición basal en los sujetos que llevaron un tratamiento real de EMTr que en los del grupo \textit{sham}.
    \item Habrá una mayor disminución en la medida de impulsividad a las dos semanas, tres y seis meses en comparación con la medición basal en los sujetos que llevaron un tratamiento real de EMTr que en los del grupo \textit{sham}.
\end{enumerate}
\subsection{Topológicas}
    \subsubsection{Basales}
    \begin{enumerate}[resume,label=Hipótesis \arabic*., left= \parindent]
        \item Los pacientes con adicción presentarán mayor hiperconectividad (\emph{fuerza} y \emph{densidad}) en sus redes de conectividad que los controles.
        \item Las redes de conectividad en pacientes adictos presentarán menores índices de eficiencia (\emph{escalar de mundo pequeño} y \emph{eficiencia local} y \emph{global}) que las de los controles.
    \end{enumerate}
    \subsubsection{Relacionadas con EMTr}
    \begin{enumerate}[resume,label=Hipótesis \arabic*., left= \parindent]
        \item La conectividad de la red (\emph{fuerza} y \emph{densidad}) disminuirá con el tratamiento real de EMTr en comparación con \textit{sham}.
        \item La eficiencia de la red (\emph{eficiencia local}, \emph{global} y \emph{mundo pequeño}) aumentará con el tratamiento real de EMTr en comparación del \textit{sham}.
    \end{enumerate}
    \subsubsection{Relacionadas con mejoría clínica}
    \begin{enumerate}[resume,label=Hipótesis \arabic*., left= \parindent]
        \item La disminución en conectividad de red estará asociada a una disminución en \textit{craving}.
        \item El aumento en eficiencia de la red estará asociado a una disminución en \textit{craving}.
        \item La disminución en conectividad de red estará asociada a una disminución en impulsividad.
        \item El aumento en eficiencia de la red estará asociado a una disminución en impulsividad.
    \end{enumerate}

